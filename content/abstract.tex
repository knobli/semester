%%%%%%%%%%%%%%%%%%%%%%%%%%%%%%%%%%%%%%%%%%%%%%%%%%%%%%%%%%%%%%%%%
%  _____   ____  _____                                          %
% |_   _| /  __||  __ \    Institute of Computitional Physics   %
%   | |  |  /   | |__) |   Zuercher Hochschule Winterthur       %
%   | |  | (    |  ___/    (University of Applied Sciences)     %
%  _| |_ |  \__ | |        8401 Winterthur, Switzerland         %
% |_____| \____||_|                                             %
%%%%%%%%%%%%%%%%%%%%%%%%%%%%%%%%%%%%%%%%%%%%%%%%%%%%%%%%%%%%%%%%%
%
% Project     : LaTeX doc Vorlage für Windows ProTeXt mit TexMakerX
% Title       : 
% File        : abstract.tex Rev. 00
% Date        : 23.4.12
% Author      : Remo Ritzmann
% Feedback bitte an Email: remo.ritzmann@pfunzle.ch
%
%%%%%%%%%%%%%%%%%%%%%%%%%%%%%%%%%%%%%%%%%%%%%%%%%%%%%%%%%%%%%%%%%

\thispagestyle{empty}



\newpage
\thispagestyle{empty}
\chapter*{Abstract}\label{abstract}
Der Turnverein Grafstal hat etwa 50 aktive Mitglieder und engagiert sich in der Leichtathletik. Der Verein verwaltet seit gut zwei Jahren seine Veranstaltungen und die dazugehörigen Anmeldungen auf einer Homepage. Im Zusammenhang mit den Feierlichkeiten des 125-jährigen Jubiläums wünschte er sich ein App.\\

Diese Semesterarbeit befasst sich mit der Erstellungen von dieser App, welche auf verschiedene Plattformen verfügbar sein sollte. Desweiteren war eine Fahrgemeinschaftsverwaltung für die Entlastung des Chats und Anmeldefunktionalität für die Veranstaltungen gewünscht.\\

Im Rahmen der Arbeit wurde die Stakeholdern evaluiert und mit diesen eine Anforderungsanalyse durchgeführt. Die Analyse brachte die wichtigsten Anfordernungen an das App hervor, desweiteren wurden Mockups für die entstehenden Ansichten erstellt. Durch eine Nutzwertanalyse wurde die geeignete Architektur für das Apps ermittelt. Ausgehend von dieser Architektur wurde dann das App entwickelt\\

Bei der Ist-Analyse des Backends fiel auf, dass dieses keine klare Strutkur aufwies, deshalb wurde ein Refactoring angedacht und durchgeführt. Im Anschluss wurde das Backend um ein RESTFul API erweitert. Das resultierende App wurde mit dem Framework Phonegap und jQuery Mobile für Android und iOS erstellt. Die Daten für die verschiedenen Seiten werden per AJAX-Request über das RESTFul API geladen. Desweiteren wurden die zwei Push-Nachrichten Dienste von Android und iOS in das App eingebunden. Eine Woche nach der Bereitstellung der App konnten bereits 50 Downloads verzeichnet werden. Die Mitglieder verwenden die Fahrgemeinschaftsverwaltung, der Chat wurde entlastet und die Anmeldungen an Veranstaltungen werden vermehrt über das App gemacht.

\chapter*{Vorwort}\label{vorwort}
Apps sind aus der heutigen Zeit nicht mehr weg zu denken. Sie schaffen neue Möglichkeiten und erleichtern unser Leben. Sie sind jedoch auch mit neuen Pflichten verbunden, es wird von einem erwartet, dass man ständig online und erreichbar ist. Es wird von grösseren Dienstleistungsunternehmen in der heutigen Zeit erwartet, ein App zu besetzten, zumal inzwischen das Smartphone für unter 40jährige wichtiger ist als der Fernseher (siehe \cite{digitalisierungsbericht2014}). An dem letzten Apple Event (siehe \cite{apple_event_sept_2014}) wurde berichtet, dass bereits über 1.3 Millonen Apps im App Store sind. Mich faszinieren Apps schon eine ganze Weile, dies war mit einer der Gründe, warum ich an einem zweiwöchigen Austauschprojekt an der Grand Valley State University in Michigan teilnahm. Das Thema in diesem Austauschprogramm war Mobile App Entwicklung.\\

Das App für den Turnverein war für mich eine gute Gelegenheit für eine Semesterarbeit. Das Projekt hat mich von Anfang an gefesselt und war auch in der vorgegebenen Zeit umsetzbar. Während und nachdem Aufenthalt in Michigan hatten ich schon einmal mit einem Entwurf einer App angefangen, also wusste ich in etwa was auf mich zu kam.\\

Mein Dank gebührt meinen Kommilitonen, allen vorweg Roman Lickel, Max Schrimpf und Dominic Schlegel, für die Hilfe in technischen Fragen und Tipps zur Dokumentation. Ganz besonderes möchte ich mich bei Beat Seeliger für die gute Betreuung und seine konstruktive Kritik und bei meinen Korrekturlesern bedanken.