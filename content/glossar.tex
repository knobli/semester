%%%%%%%%%%%%%%%%%%%%%%%%%%%%%%%%%%%%%%%%%%%%%%%%%%%%%%%%%%%%%%%%%
%
% Project     : Turnverein App
% Title       : 
% File        : glossar.tex Rev. 00
% Date        : 07.07.14
% Author      : Raffael Santschi
%
%%%%%%%%%%%%%%%%%%%%%%%%%%%%%%%%%%%%%%%%%%%%%%%%%%%%%%%%%%%%%%%%%

\chapter*{Glossar}\label{chap.glossar}
\addcontentsline{toc}{section}{Glossar}\label{cha:glossar}

In diesem Abschnitt werden Abkürzungen und Begriffe kurz erklärt.

\begin{longtable}{|m{3cm}|m{11cm}|}\hline	
	\rowcolor{gray} \textbf{Abk}&
	Abkürzung \\ \hline		

	\textbf{APNS}&
	(siehe Apple Push Notification Service)\\ \hline	

	\textbf{Apple Push Notification Service}&
	Der Service von Apple, welcher für das Versenden von Push-Nachrichten auf iOS Geräte angesprochen wird\\ \hline	

	\textbf{Ajax}&
	(siehe Asynchronous JavaScript and XML)\\ \hline	

	\textbf{Asynchronous JavaScript and XML}&
	Ajax ... bezeichnet ein Konzept der asynchronen Datenübertragung zwischen einem Browser und dem Server. Dieses ermöglicht es, HTTP-Anfragen durchzuführen, während eine HTML-Seite angezeigt wird, und die Seite zu verändern, ohne sie komplett neu zu laden.\cite{wiki_ajax}\\ \hline	

	\textbf{Basisfaktor}&
	Basisfaktoren (unterbewusste Anforderungen) muss das System in jedem Fall vollständig erfüllen, sonst stellt sich beim Stakeholder massive Unzufriedenheit ein. \cite{req_eng_book}\\ \hline	

	\textbf{Begeisterungsfaktor}&
	Begeisterungsfaktoren (unbewusste Anforderungen) sind Merkmale eines Systems, deren Wert ein Stakeholder erst erkennt, wenn er sie selbst ausprobieren kann oder sie vom Requirements Engineer vorgeschlagen werden.\cite{req_eng_book}\\ \hline	

	\textbf{Büchsenliste}&
	Die Büchsenliste wurde vom Turnverein eingeführt, um die Mitgliedern anzuspornen. Ziel ist es bei jedem Wettkampf mindestens gleich gut, wie im letzten Wettkampf zu sein. In der Liste werden die letzten Resultate aufgeführt, damit man weiss, was das minimale Ziel ist. Falls ein Mitglied ein schlechteres Resultat erzielt, muss er einen vordefinierten Betrag in die Büchse (daher der Name) zahlen.\\ \hline	

	\textbf{Entity}&
	Entity (auch Entität) ist ein eindeutig zu bestimmendes Daten-Objekt \\ \hline	

	\textbf{GCM}&
	(siehe Google Cloud Messaging for Android)\\ \hline

	\textbf{Google Cloud Messaging for Android}&
	Der Service von Google, welcher für das Versenden von Push-Nachrichten auf Android Geräte angesprochen wird\\ \hline	

	\textbf{Integrated Development Environment}&
	Eine integrierte Entwicklerumgebung (englisch Integrated Development Environment) beinhalten einen Texteditor, Compiler (falls dieser benötigt wird), Debugger, Formatierungsfunktionen und in dem Kontext der Appentwicklung die Möglichkeit der Erstellung von grafischen Benutzeroberflächen.\\ \hline

	\textbf{IDE}&
	(siehe Integrated Development Environment)\\ \hline

	\textbf{JavaScript Object Notation}&
	Die JavaScript Object Notation, kurz JSON ..., ist ein kompaktes Datenformat in einer einfach lesbaren Textform zum Zweck des Datenaustauschs zwischen Anwendungen.\cite{wiki_json}\\ \hline

	\textbf{JSON}&
	(siehe JavaScript Object Notation)\\ \hline

	\textbf{localStorage}&
	localStorage ist ein begrenzter Speicherbereich im Document Object Model (DOM), welcher auch nach dem Schliessen des Browsers ausgelesen werden kann\\ \hline	

	\textbf{Mockup}&
	Der aus dem Englischen stammende Begriff Mock-up oder Mockup (auch Maquette) bezeichnet im Deutschen beispielsweise eine Attrappe. Er wird heute aber meist für ein maßstäblich gefertigtes Modell bzw. eine Nachbildung zu Präsentationzwecken benutzt, demgegenüber ist der Prototyp ein funktionsfähiges Modell.\cite{wiki_mockup}\\ \hline	

	\textbf{Objektrelationale Abbildung}&
	Objektrelationale Abbildung (englisch object-relational mapping, ORM) ist eine Technik der Softwareentwicklung, mit der ein in einer objektorientierten Programmiersprache geschriebenes Anwendungsprogramm seine Objekte in einer relationalen Datenbank ablegen kann. Dem Programm erscheint die Datenbank dann als objektorientierte Datenbank, was die Programmierung erleichtert. \cite{wiki_orm}\\ \hline	

	\textbf{Object-relational mapping}&
	(siehe Objektrelationale Abbildung)\\ \hline	

	\textbf{Object Query Language}&
	Object Query Language (OQL) ist stark an SQL angeleht, wobei man nicht mit den Spalten der Tabelle, sondern mit den Attributen des Objekts Abfragen erstellt.\\ \hline	

	\textbf{OQL}&
	(siehe Object Query Language)\\ \hline

	\textbf{ORM}&
	(siehe Object-relational mapping)\\ \hline

	\textbf{PHP}&
	PHP (rekursives Akronym und Backronym für „PHP: Hypertext Preprocessor“, ursprünglich „Personal Home Page Tools“) ist eine Skriptsprache mit einer an C und Perl angelehnten Syntax, die hauptsächlich zur Erstellung dynamischer Webseiten oder Webanwendungen verwendet wird.\cite{wiki_php}\\ \hline	

	\textbf{Projektstrukturplan}&
	Ein Synonym für Work Breakdown Structure (WBS): Eine in der Regel an den Liefergegenständen orientierte Anordnung von Projektelementen, die den Gesamtinhalt und -umfang des Projekts strukturiert und definiert.\cite{proj_mgmt_book}\\ \hline	

	\textbf{Refactoring}&
	Eine Veränderung in der internen Strukturen der Software, welche sie verständlicher und wartbarer macht, jedoch ohne eine Änderung des ursprünglichen Verhaltens.\cite{feathers2004working}\\ \hline		

	\textbf{Stakeholder}&
	Stakeholder sind für den Requirement Engineer wichtige Quellen zur Identifikation möglicher Anforderungen des Systems.\cite{req_eng_book}\\ \hline		

	\textbf{Session-Token}&
	Eine Session-ID ist eine eindeutige ID, welche bei der Anmeldung generiert wird, sie kann dann zur Authentifizierung verwendet werden. Eine Session-ID ist nur eine gewisse Zeit gültig \\ \hline		

\end{longtable}
