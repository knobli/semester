%%%%%%%%%%%%%%%%%%%%%%%%%%%%%%%%%%%%%%%%%%%%%%%%%%%%%%%%%%%%%%%%%
%  _____   ____  _____                                          %
% |_   _| /  __||  __ \    Institute of Computitional Physics   %
%   | |  |  /   | |__) |   Zuercher Hochschule Winterthur       %
%   | |  | (    |  ___/    (University of Applied Sciences)     %
%  _| |_ |  \__ | |        8401 Winterthur, Switzerland         %
% |_____| \____||_|                                             %
%%%%%%%%%%%%%%%%%%%%%%%%%%%%%%%%%%%%%%%%%%%%%%%%%%%%%%%%%%%%%%%%%
%
% Project     : LaTeX doc Vorlage für Windows ProTeXt mit TexMakerX
% Title       : 
% File        : diskussion.tex Rev. 00
% Date        : 7.5.12
% Author      : Remo Ritzmann
% Feedback bitte an Email: remo.ritzmann@pfunzle.ch
%
%%%%%%%%%%%%%%%%%%%%%%%%%%%%%%%%%%%%%%%%%%%%%%%%%%%%%%%%%%%%%%%%%

\chapter{Schlussfolgerungen}\label{chap.Schlussfolgerungen}
In der Schlussfolderung wird kurz auf die Verwendung des Produktes, das Fazit des Erfassers und den Ausblick eingegangen.

\section{Verwendung}\label{fazit_verwendung}
Das App fand schnell grossen Anklang, die Neuigkeit über das App verbreitete sich automatisch. Die Turnvereine machten die Öffentlichkeit zusätzlich über Facebook und ihre Homepage auf das App aufmerksam. Eine Woche nach der Bekanntmachung hatten bereits 50 Personen das App heruntergeladen. Die Fahrgemeinschaftsverwaltung wird benutzt und die Anmeldungen an Veranstaltungen werden vermehrt über das App gemacht.

\section{Fazit}\label{fazit}

Die Entwicklung einer App hat sehr viel Spass gemacht und die Zeit verging wie im Flug. Ich fand es toll das angeignete Wissen über die Anforderungsanalyse und Architektur einsetzen zu können. Die klar definierten Anforderungen haben mir bei der Entwicklung geholfen und liesen nicht viel Spielraum für Missverständnisse. Die enge Zusammenarbeit mit den Mitgliedern war mir bei der Implementation ebenfalls eine grosse Hilfe.\\

Phonegap ist ein gutes Framework, welches einem viel Arbeit abnimmt und in diesem Projekt die richtige Entscheidung war. Ein App mit Phonegap zu entwicklen, lohnt sich bereits ab zwei Plattformen, da der Wartungsaufwand sonst enorm gross wird. Ich fand es auch toll, dass es auf HTML und Javascript basiert, was mir den Einstieg erleichtert. jQuery Mobile ist zwar mächtig, aber nur so lange man die Standardelemente und Funktionen verwendet, sonst kann es sehr schnell in viel Arbeit ausarten. Die Wahl jQuery zu verwenden, war eine gute und effiziente Lösung für diesen Prototyp, ich werde mir jedoch nochmals überlegen, ob ich vielleicht nicht auf ein flexibleres Framework wechseln möchte.\\

Das Refactoring war in diesem Projekt dringend nötig, hat jedoch auch viel mehr Zeit gekostet als geplant. Ich wünschte mir für spätere Projekte, dass ich bereits Tests vorfinde, was den Umbau einiges einfacher gemacht hätte. Es war spannend eine grössere Datenbankmigration auf produktiven Daten zu planen und dann auch durchzuführen. Mit der neuen Testumgebung und dem ORM machte das Entwickeln richtig Spass.\\

\section{Ausblick}\label{fazit_ausblick}
Ich werde in einem weiteren Schritt wahrscheindlich das App mit Angular JS erweitern. Es sind bereits viele neue Anforderungen von den Mitgliedern eingetroffen, welche in den kommenden Monaten umgesetzt werden. Das positiven Feedback der Mitglieder haben mich sehr gefreut und auch motiviert. Die App ist für den Turnverein, zu mindest für die jüngere Generation, ein voller Erfolg und so soll es auch sein. Zum Schluss kann ich nur noch sagen, es ist ein tolles Gefühl seine eigene App im App Store zu sehen!