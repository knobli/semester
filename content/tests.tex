%%%%%%%%%%%%%%%%%%%%%%%%%%%%%%%%%%%%%%%%%%%%%%%%%%%%%%%%%%%%%%%%%
%  _____   ____  _____                                          %
% |_   _| /  __||  __ \    Institute of Computitional Physics   %
%   | |  |  /   | |__) |   Zuercher Hochschule Winterthur       %
%   | |  | (    |  ___/    (University of Applied Sciences)     %
%  _| |_ |  \__ | |        8401 Winterthur, Switzerland         %
% |_____| \____||_|                                             %
%%%%%%%%%%%%%%%%%%%%%%%%%%%%%%%%%%%%%%%%%%%%%%%%%%%%%%%%%%%%%%%%%
%
% Project     : LaTeX doc Vorlage für Windows ProTeXt mit TexMakerX
% Title       : 
% File        : resultate.tex Rev. 00
% Date        : 23.4.12
% Author      : Remo Ritzmann
% Feedback bitte an Email: remo.ritzmann@pfunzle.ch
%
%%%%%%%%%%%%%%%%%%%%%%%%%%%%%%%%%%%%%%%%%%%%%%%%%%%%%%%%%%%%%%%%%

\chapter{Tests}\label{chap.tests} 
In diesem Kapitel wird auf die verschiedenen Tests und Varianten, welche in diesem Projekt verwendet wurden, eingegangen.

\section{Einführung}
In diesem Projekt hat man die sieben Grunsätze aus \cite{test_soft_book} als Leitlinie beim Testen genommen:
\begin{enumerate}
\item \textbf{Testen zeigt die Anwesenheit von Fehlern}: Testabdeckung mit Sonar ermittelt und Anforderungen überprüft
\item \textbf{Vollständiges Testen ist nicht möglich}: Es wurde solange getestet bis die Fehlerfindungsrate einen bestimmte Punkt unterschritten hatte
\item \textbf{Mit dem Testen frühzeitig beginnen}: Die Tests für die neuen Funktionen wurden zur gleiche Zeit, wie der Code, geschrieben
\item \textbf{Häufung von Fehlern}: Wenn Fehler auftratten, dann wurde diese Komponente nach der Behebung nochmals intensiv getestet
\item \textbf{Zunehmende Testresistenz (Pesticide paradoc)}: Als Änderungen am Code gemacht wurden, wurden die Tests erweitert bzw. angepasst
\item \textbf{Testen ist abhängig vom Umfeld}: Dieses System ist nicht so sicherheitskritische, wie eine Bank, jedoch werden die Dienste rege bentutzt, somit sollte die Testintensität in einem normalen Rahmen liegen.
\item \textbf{Trugschluss: Keine Fehler bedeutet eine brauchbares System}: Die Stakeholder wurden schon bei der Mock-up Erstellung mit einbezogen und bekamen auch während der Entwicklung immer wieder einen Blick in den aktuellen Stand.
\end{enumerate}

\section{Testing}
Komponententests und Integrationstests wurden im Backend mit PHPUnit durchgeführt. Bei den Integrationstests wurde eine In-Memory Datenbank verwendet, welche bei jedem Testlauf neugeladen wird, somit ist man komplett unabhängig von äusseren Einflüssen. Die Testabdeckung wurde mit Sonar überprüft, hier wurde vorallem ein Augenmerk auf die neuen Funktionen und in diesem Projekt verwendete Funktionen gelegt. Neben den verschiedenen Entitäten wurden die Repositories und die Services getestet. (siehe \cite{test_soft_book})

\subsection{ Testphasen}
Es wurde festgelegt, dass das Projekt nicht mit der Implementation fertig ist, sondern mit der definitiven Auslieferung. Damit der Kunde genug Erfahrungen sammeln kann, wurden Testphasen definiert und erst nach Beendigung dieser Testphasen, konnte das Produkt abgenommen werden. Die Testphasen wurden getrennt, damit Fehler klar abgegrenzt werden konnten und für die gemeldeten Fehler auch mehr Zeit zur Verfügung stand.

\subsubsection{Backend}
Die Testphase des Backends begann nachdem lokalen Testing des Refactorings und der produktiv Schaltung des Backends. Die Mitglieder wurden informiert, dass sie theoretisch keine grossen Änderungen feststellen sollten und sie sollten sich doch bitte bei negativen Abweichungen zur vorherigen Version sofort melden. Es wurden nur wenige Speziallfälle gemeldet, welche schnell behoben werden konnten.

\subsubsection{App}
Da das Freigeben der iOS App mehrere Tage benötigt, wurde die erste Testphase auf Android gemacht, damit man schneller reagieren konnte. Das App wurde hochgeladen und nur einigen Personen davon erzählt, nach kurzer Zeit gab es bereits über 10 Downloads. Das Feedback war gut und nur einige kleine Fehler wurden gefunden. Alle Fehler und ein paar kleine Verbesserungsvorschläge wurden sofort umgesetzt, getest und dann eine neue Version hochgeladen. Grössere Anforderungen wurden erfasst und auf die kommenden Versionen verschoben. Als keine negativen Rückmeldungen mehr kamen und auch beim Nachfragen nichts mehr zum Vorschein kam, wurde die Testphase mit dem Einliefern des iOS Apps beendet.

\section{System- und Abnahmetest}
Nach erfolgreichem Testing und abschluss der Testphasen, wurde für den Abnahmetest ein Testprotokoll erstellt, welches vorgängig im Systemtest schon durchgeganen wurde. Das Testprotokoll wurde danach dem Kunden übergeben und von diesem selber nochmals abgearbeitet.

\subsection{Testprotokoll}
Das Testprotokoll basiert auf den Use Cases (siehe \ref{use_cases}) und den Anforderungen (siehe \ref{anforderungen}). Akzeptanz Kriterien mit einen UND- oder ODER-Verknüfung wurden in aufgesplitet, um sicher zu gehen, dass beide Bedingungen erfüllt sind.

\begin{longtable}{ l | p{7cm} | l | l }

	\hline
	\rowcolor{gray}
	ID	&	Test			&	Herkunft			&	erfüllt / nicht erfüllt\\ \hline
	1a	&	Bei bekanntem Username und Password muss es dem Mitglied möglich sein, sich beim Backend anzumelden			
				&	\nameref{table:req_1} 	&	erfüllt\\ \hline
	1b	&	und dann erweiterte Möglichkeiten und Informationen zu erhalten.
				&	\nameref{table:req_1} 	&	erfüllt\\ \hline
	2a	&	Falls ein unbekannter Username eingegeben wurden kommt eine entsprechende Fehlermeldung.			
				&	\nameref{table:req_1} 	&	erfüllt\\ \hline
	2b	&	Falls ein falsche Passwort eingegeben wurden kommt eine entsprechende Fehlermeldung.			
				&	\nameref{table:req_1} 	&	erfüllt\\ \hline
	3	&	Das angemeldete Mitglied kann eine Fahrgemeinschaft eröffnen.		
				&	\nameref{table:req_2} 	&	erfüllt\\ \hline
	4a	&	Bei ungültigen Informationen kommt eine entsprechende Fehlermeldung
				&	\nameref{table:req_2} 	&	nicht erfüllt\\ \hline
	4b	&	Bei nicht ausreichenden Informationen kommt eine entsprechende Fehlermeldung
				&	\nameref{table:req_2} 	&	erfüllt\\ \hline
	5	&	Das angemeldete Mitglied kann die Fahrgemeinschaften anzeigen lassen
				&	\nameref{table:req_3} 	&	nicht erfüllt\\ \hline
	6	&	Das angemeldete Mitglied sieht einen Hinweis, falls es keine Fahrgemeinschaften gibt
				&	\nameref{table:req_3} 	&	erfüllt\\ \hline
	7	&	Das angemeldete Mitglied kann einer Fahrgemeinschaft beitreten
				&	\nameref{table:req_4} 	&	erfüllt\\ \hline
	8	&	Das angemeldete Mitglied kann einer Fahrgemeinschaft nicht erneut beitreten
				&	\nameref{table:req_4} 	&	erfüllt\\ \hline
	9	&	Das angemeldete Mitglied erhält eine Fehlermeldung falls die Fahrgemeinschaft keine Kapazität mehr aufweisst
				&	\nameref{table:req_4} 	&	plz test\\ \hline
	10	&	Das angemeldete Mitglied kann sich für einen Anlass anmelden
				&	\nameref{table:req_5} 	&	erfüllt\\ \hline
	11	&	Das angemeldete Mitglied kann sich für einen Anlass nicht erneut anmelden
				&	\nameref{table:req_5} 	&	plz test\\ \hline
	12	&	Der Benutzer erhält informationen über den ausgewählten Anlass
				&	\nameref{table:req_6} 	&	erfüllt\\ \hline
	13	&	Der Benutzer sieht einen Hinweis, falls keine Anlässe vorhanden sind
				&	\nameref{table:req_6} 	&	erfüllt\\ \hline
	14	&	Das angemeldete Mitglied sieht die erweiterten Informationen zum Anlass
				&	\nameref{table:req_7} 	&	erfüllt\\ \hline
	15	&	Nicht angemeldete Mitglieder sehen die erweiterten Informationen zum Anlass nicht
				&	\nameref{table:req_7} 	&	erfüllt\\ \hline
	16a	&	Der Benutzer erhält Informationen über die verschiedenen Vereine
				&	\nameref{table:req_8} 	&	erfüllt\\ \hline
	16b	&	Der Benutzer erhält Informationen über die verschiedenen Riegen
				&	\nameref{table:req_8} 	&	erfüllt\\ \hline
	17	&	Der Benutzer kann die aktuellen Berichte lesen
				&	\nameref{table:req_9} 	&	erfüllt\\ \hline
	18	&	Das angemeldete Mitglied erhält die Push-Nachrichten
				&	\nameref{table:req_10} 	&	erfüllt\\ \hline
	19a	&	Ein Nutzer mit einem Android Mobiltelefon kann die App öffnen
				&	\nameref{table:req_nf_1} 	&	erfüllt\\ \hline
	19b	&	Ein Nutzer mit einem Android Mobiltelefon kann die App verwenden
				&	\nameref{table:req_nf_1} 	&	erfüllt\\ \hline
	20a	&	Ein Nutzer mit einem iPhone kann die App öffnen
				&	\nameref{table:req_nf_2} 	&	erfüllt\\ \hline
	20b	&	Ein Nutzer mit einem iPhone kann die App verwenden
				&	\nameref{table:req_nf_2} 	&	erfüllt

\end{longtable}

\newpage
\section{Abnahme Protokoll}
Nach dem erfolgreichen Abnahmetest wurde vom Kunde das Abnahme Protokoll unterzeichnet.
\begin{figure}[h]
\centering
\includegraphics[scale=0.55]{images/word/Abnahmeprotokoll.pdf}
\caption{Abnahmeprotokoll}
\label{fig:abnahmeprotokoll}
\end{figure}
