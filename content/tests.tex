%%%%%%%%%%%%%%%%%%%%%%%%%%%%%%%%%%%%%%%%%%%%%%%%%%%%%%%%%%%%%%%%%
%
% Project     : Turnverein App
% Title       : 
% File        : tests.tex Rev. 00
% Date        : 07.07.14
% Author      : Raffael Santschi
%
%%%%%%%%%%%%%%%%%%%%%%%%%%%%%%%%%%%%%%%%%%%%%%%%%%%%%%%%%%%%%%%%%

\chapter{Tests}\label{chap.tests} 
In diesem Kapitel wird auf die verschiedenen Tests und Varianten, welche in diesem Projekt verwendet wurden, eingegangen.

\section{Einführung}
In diesem Projekt hat man die sieben Grundsätze aus \cite{test_soft_book} als Leitlinie beim Testen genommen:
\begin{enumerate}
\item \textbf{Testen zeigt die Anwesenheit von Fehlern}: Testabdeckung mit Sonar ermittelt und Anforderungen überprüft
\item \textbf{Vollständiges Testen ist nicht möglich}: Es wurde solange getestet bis die Fehlerfindungsrate einen bestimmte Punkt unterschritten hatte
\item \textbf{Mit dem Testen frühzeitig beginnen}: Die Tests für die neuen Funktionen wurden zur gleichen Zeit wie der Code, geschrieben
\item \textbf{Häufung von Fehlern}: Wenn Fehler auftraten, dann wurde diese Komponente nach der Behebung nochmals intensiv getestet
\item \textbf{Zunehmende Testresistenz (Pesticide paradox)}: Als Änderungen am Code gemacht wurden, wurden die Tests erweitert bzw. angepasst
\item \textbf{Testen ist abhängig vom Umfeld}: Dieses System ist nicht so sicherheitskritisch, wie etwa Systeme im Bank Umfeld, die Dienste werden jedoch rege benutzt, somit sollte die Testintensität in einem normalen Rahmen liegen.
\item \textbf{Trugschluss: Keine Fehler bedeutet eine brauchbares System}: Die Stakeholder wurden schon bei der Mock-up Erstellung mit einbezogen und bekamen auch während der Entwicklung immer wieder einen Blick in den aktuellen Stand.
\end{enumerate}

\section{Testing}
Komponententests und Integrationstests wurden im Backend mit PHPUnit durchgeführt. Bei den Integrationstests wurde eine In-Memory Datenbank verwendet, welche bei jedem Testlauf neu geladen wird, somit ist man komplett unabhängig von äusseren Einflüssen. Die Testabdeckung wurde mit Sonar überprüft, hier wurde vor allem ein Augenmerk auf die neuen Funktionen und in diesem Projekt verwendete Funktionen gelegt. Neben den verschiedenen Entitäten wurden die Repositories und die Services getestet. (siehe \cite{test_soft_book})

\subsection{Testphasen}
Es wurde festgelegt, dass das Projekt nicht mit der Implementation fertig ist, sondern mit der definitiven Auslieferung. Damit der Kunde genug Erfahrungen sammeln kann, wurden Testphasen definiert. Erst nach Beendigung dieser Testphasen, konnte das Produkt abgenommen werden. Die Testphasen wurden getrennt, damit Fehler klar abgegrenzt werden konnten und für die Behandlung der gemeldeten Fehler auch mehr Zeit zur Verfügung stand.

\subsubsection{Backend}
Die Testphase des Backends begann nach dem lokalen Testing des Refactorings und der Produktiv Schaltung des Backends. Die Mitglieder wurden informiert, dass sie theoretisch keine grossen Änderungen feststellen sollten und sie sollten sich doch bitte bei negativen Abweichungen zur vorherigen Version sofort melden. Es wurden nur wenige Spezialfälle gemeldet, welche schnell behoben werden konnten.

\subsubsection{App}
Da die Freigabe der iOS App mehrere Tage benötigt, wurde die erste Testphase auf Android gemacht. So konnte schneller Feedback eingeholt werden. Die App wurde hochgeladen und nur wenige Personen wurden informiert. Nach wenigen Stunden gab es bereits über 10 Downloads. Das Feedback war gut, lediglich einige kleine Fehler wurden gefunden. Alle Fehler wurden behoben und ein paar kleine Verbesserungsvorschläge umgesetzt, getestet und dann in einer neuen Version hochgeladen. Grössere Anforderungen wurden erfasst, deren Umsetzung jedoch auf kommende Versionen verschoben. Als keine negativen Rückmeldungen mehr kamen und auch beim Nachfragen nichts mehr zum Vorschein kam, wurde die Testphase mit dem Hochladen der iOS App beendet.

\section{System- und Abnahmetest}
Nach erfolgreichem Testing und Abschluss der Testphasen wurde für den Abnahmetest ein Testprotokoll erstellt, welches vorgängig im Systemtest schon durchgegangen wurde. Das Testprotokoll wurde danach dem Kunden übergeben und von diesem selber nochmals abgearbeitet.

\subsection{Testprotokoll}
Das Testprotokoll basiert auf den Use Cases (siehe \ref{use_cases}) und den Anforderungen (siehe \ref{anforderungen}). Akzeptanz Kriterien mit UND- oder ODER-Verknüpfung wurden aufgesplittet, um sicher zu gehen, dass beide Bedingungen erfüllt sind.

%%\begin{longtable}{ l | p{7cm} | l | l }
\begin{longtable}{>{\raggedright}m{1cm}m{7cm}m{3cm}m{2cm}}

\caption[Testprotokoll]{\label{table:tests}Testprotokoll}\\ 
\toprule
\textbf{ID}&\textbf{Test}&\textbf{Herkunft}&\textbf{erfüllt / nicht erfüllt}\\ \midrule\addlinespace
\endfirsthead
\caption*{\textbf{Tabelle~\ref{table:tests} (Fortsetzung):} Testprotokoll}\\ \toprule
\textbf{ID}&\textbf{Test}&\textbf{Herkunft}&\textbf{erfüllt / nicht erfüllt}\\ \midrule\addlinespace
\endhead

\bottomrule\multicolumn{2}{>{\small\raggedleft\arraybackslash}r}{\slshape Fortsetzung auf der nächsten Seite}\\
\endfoot
\bottomrule
\endlastfoot	

	\addlinespace
	1a	&	Bei bekanntem Username und Password muss es dem Mitglied möglich sein, sich beim Backend anzumelden			
				&	\nameref{table:req_1} 	&	erfüllt\\ \addlinespace\hline \addlinespace
	1b	&	Nach erfolgreichem Anmelden muss es dem Mitglied möglich sein erweiterte Möglichkeiten und Informationen zu erhalten.
				&	\nameref{table:req_1} 	&	erfüllt\\ \addlinespace\hline \addlinespace
	2a	&	Falls ein unbekannter Username eingegeben wurde kommt eine entsprechende Fehlermeldung.			
				&	\nameref{table:req_1} 	&	erfüllt\\ \addlinespace\hline \addlinespace
	2b	&	Falls ein falsches Passwort eingegeben wurde kommt eine entsprechende Fehlermeldung.			
				&	\nameref{table:req_1} 	&	erfüllt\\ \addlinespace\hline \addlinespace
	3	&	Das angemeldete Mitglied kann eine Fahrgemeinschaft eröffnen.		
				&	\nameref{table:req_2} 	&	erfüllt\\ \addlinespace\hline \addlinespace
	4a	&	Bei ungültigen Informationen kommt eine entsprechende Fehlermeldung
				&	\nameref{table:req_2} 	&	erfüllt\\ \addlinespace\hline \addlinespace
	4b	&	Bei nicht ausreichenden Informationen kommt eine entsprechende Fehlermeldung
				&	\nameref{table:req_2} 	&	erfüllt\\ \addlinespace\hline \addlinespace
	5	&	Das angemeldete Mitglied kann die Fahrgemeinschaften anzeigen lassen
				&	\nameref{table:req_3} 	&	erfüllt\\ \addlinespace\hline \addlinespace
	6	&	Das angemeldete Mitglied sieht einen Hinweis, falls es keine Fahrgemeinschaften gibt
				&	\nameref{table:req_3} 	&	erfüllt\\ \addlinespace\hline \addlinespace
	7	&	Das angemeldete Mitglied kann einer Fahrgemeinschaft beitreten
				&	\nameref{table:req_4} 	&	erfüllt\\ \addlinespace\hline \addlinespace
	8	&	Das angemeldete Mitglied kann einer Fahrgemeinschaft nicht erneut beitreten
				&	\nameref{table:req_4} 	&	erfüllt\\ \addlinespace\hline \addlinespace
	9	&	Das angemeldete Mitglied erhält eine Fehlermeldung falls die Fahrgemeinschaft keine Kapazität mehr aufweist
				&	\nameref{table:req_4} 	&	erfüllt\\ \addlinespace\hline \addlinespace
	10	&	Das angemeldete Mitglied kann sich für einen Anlass anmelden
				&	\nameref{table:req_5} 	&	erfüllt\\ \addlinespace\hline \addlinespace
	11	&	Das angemeldete Mitglied kann sich für einen Anlass nicht erneut anmelden
				&	\nameref{table:req_5} 	&	erfüllt\\ \addlinespace\hline \addlinespace
	12	&	Der Benutzer erhält informationen über den ausgewählten Anlass
				&	\nameref{table:req_6} 	&	erfüllt\\ \addlinespace\hline \addlinespace
	13	&	Der Benutzer sieht einen Hinweis, falls keine Anlässe vorhanden sind
				&	\nameref{table:req_6} 	&	erfüllt\\ \addlinespace\hline \addlinespace
	14	&	Das angemeldete Mitglied sieht die erweiterten Informationen zum Anlass
				&	\nameref{table:req_7} 	&	erfüllt\\ \addlinespace\hline \addlinespace
	15	&	Nicht angemeldete Mitglieder sehen die erweiterten Informationen zum Anlass nicht
				&	\nameref{table:req_7} 	&	erfüllt\\ \addlinespace\hline \addlinespace
	16a	&	Der Benutzer erhält Informationen über die verschiedenen Vereine
				&	\nameref{table:req_8} 	&	erfüllt\\ \addlinespace\hline \addlinespace
	16b	&	Der Benutzer erhält Informationen über die verschiedenen Riegen
				&	\nameref{table:req_8} 	&	erfüllt\\ \addlinespace\hline \addlinespace
	17	&	Der Benutzer kann die aktuellen Berichte lesen
				&	\nameref{table:req_9} 	&	erfüllt\\ \addlinespace\hline \addlinespace
	18	&	Das angemeldete Mitglied erhält die Push-Nachrichten
				&	\nameref{table:req_10} 	&	erfüllt\\ \addlinespace\hline \addlinespace
	19a	&	Ein Nutzer mit einem Android Mobiltelefon kann die App öffnen
				&	\nameref{table:req_nf_1} 	&	erfüllt\\ \addlinespace\hline \addlinespace
	19b	&	Ein Nutzer mit einem Android Mobiltelefon kann die App verwenden
				&	\nameref{table:req_nf_1} 	&	erfüllt\\ \addlinespace\hline \addlinespace
	20a	&	Ein Nutzer mit einem iPhone kann die App öffnen
				&	\nameref{table:req_nf_2} 	&	erfüllt\\ \addlinespace\hline \addlinespace
	20b	&	Ein Nutzer mit einem iPhone kann die App verwenden
				&	\nameref{table:req_nf_2} 	&	erfüllt\\ \addlinespace\hline \addlinespace
	21	&	Das System zeigt 'keine Internetverbindung vorhanden' an
				&	Use Case UC-1, UC-2, UC-4, UC-5
										&	erfüllt

\end{longtable}

\section{Abnahme Protokoll}
Nach dem erfolgreichen Abnahmetest wurde vom Kunden das Abnahme Protokoll unterzeichnet (siehe Anhang \ref{anhang_abnahmeprotokoll}).
